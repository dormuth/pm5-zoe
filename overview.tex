Many people have limited time to travel and take a vacation. Popular travel destinations include the National Parks in the United States. A common problem is over-crowding during certain times of the year for many National Parks. It takes away from the natural surroundings of the parks, make activities dangerous and take a toll on the environment.

The goal of this project is to create a visualization that will find the optimal time of year to visit each National Park based on the average number of visitors, weather, cost and type of activity the user would like to do. The user should be able to visit any National Park at the optimal time and avoid crowds while still being able to have a good experience. This visualization problem will combine multiple data sets into one visualization that will provide an overall description of the experience one would expect at a National Park at a specific time of year.

This problem is important because it will allow the user to have better experiences at National Parks while possibly mitigating part of the over-crowding problem the parks are facing. This will entail help preserve the public lands. An decline of accidents may also be seen if this visualization helped mitigate the over-crowding issue. The National Park Service could potentially use this visualization to implement procedures to mitigate the over-crowding issue for certain times of the year. 

% You may want to end this section by summarizing it again as a list:
The work completed in this milestone included:
\begin{itemize}
  \item Added missing parts of visualization that were not completed in Milestone 4
  \item Considered suggestions from presentation feedback and incorporated ones that were simple to implement
  \item Designed evaluation
  \item Conducted evaluation on two participants
  \item Summarized results from evaluation and presentation feedback
  \item Concluded project and stated limitations and future work and evaluations
\end{itemize}