\section{Project Summary}

This project is an exploratory analysis of the National Park's in the United States. The National Park's data about number of visitors per month to each National Park, the average high and low temperatures per month, the cost of visiting the National Park, the activities available, the description and location of the National Park's were collected and encoded in a visualization. The project was selected and evaluated to be a visualization problem. Background research, previous visualizations and existing data on the problem were researched and analyzed. A list of tasks using Schneiderman's task taxonomy was used to provide the task and data abstraction. This allowed this specific problem to be generalized to fit existing visualization design techniques. The data was collected from multiple sources and put into one file. This file is the only file to have the specific collection of data from multiple sources. The visualization design was coded using D3 js and Leaflet, a library to encode the geospatial data. The project switched from a recommendation system to an exploratory visualization after the visual design encodings were mapped out by hand. The recommendation system would have required data analysis out of the scope of this project. The visualization was coded for three milestones. An evaluation was conducted using interviews as well as the questionnaire collected after the in-class presentation. These allows the design to be improved with feedback and understand which design techniques solved which tasks the best. Every milestone included a small design iteration in the way the visualization was encoded or a large one for the switch from a recommendation system to an exploratory visualization. 