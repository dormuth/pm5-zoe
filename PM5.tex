\documentclass[journal]{vgtc}                % final (journal style)
%\documentclass[review,journal]{vgtc}         % review (journal style)
%\documentclass[widereview]{vgtc}             % wide-spaced review
%\documentclass[preprint,journal]{vgtc}       % preprint (journal style)

%% Uncomment one of the lines above depending on where your paper is
%% in the conference process. ``review'' and ``widereview'' are for review
%% submission, ``preprint'' is for pre-publication, and the final version
%% doesn't use a specific qualifier.

%% These few lines make a distinction between latex and pdflatex calls and they
%% bring in essential packages for graphics and font handling.
%% Note that due to the \DeclareGraphicsExtensions{} call it is no longer necessary
%% to provide the the path and extension of a graphics file:
%% \includegraphics{diamondrule} is completely sufficient.
%%
\ifpdf%                                % if we use pdflatex
  \pdfoutput=1\relax                   % create PDFs from pdfLaTeX
  \pdfcompresslevel=9                  % PDF Compression
  \pdfoptionpdfminorversion=7          % create PDF 1.7
  \ExecuteOptions{pdftex}
  \usepackage{graphicx}                % allow us to embed graphics files
  \DeclareGraphicsExtensions{.pdf,.png,.jpg,.jpeg} % for pdflatex we expect .pdf, .png, or .jpg files
\else%                                 % else we use pure latex
  \ExecuteOptions{dvips}
  \usepackage{graphicx}                % allow us to embed graphics files
  \DeclareGraphicsExtensions{.eps}     % for pure latex we expect eps files
\fi%

%% it is recomended to use ``\autoref{sec:bla}'' instead of ``Fig.~\ref{sec:bla}''
\graphicspath{{figures/}{pictures/}{images/}{./}} % where to search for the images

\usepackage{microtype}                 % use micro-typography (slightly more compact, better to read)
\PassOptionsToPackage{warn}{textcomp}  % to address font issues with \textrightarrow
\usepackage{textcomp}                  % use better special symbols
\usepackage{mathptmx}                  % use matching math font
\usepackage{times}                     % we use Times as the main font
\renewcommand*\ttdefault{txtt}         % a nicer typewriter font
\usepackage{cite}

%% If you are submitting a paper to a conference for review with a double
%% blind reviewing process, please replace the value ``0'' below with your
%% OnlineID. Otherwise, you may safely leave it at ``0''.
\onlineid{0}

%% declare the category of your paper, only shown in review mode
\vgtccategory{Research}

%% Paper title - 1 pt for descriptive title
\title{The Optimal Time of Year to Visit Each National Park}

%% This is how authors are specified in the journal style

%% indicate IEEE Member or Student Member in form indicated below
%% 1 pt for name
\author{Zoe Dormuth}
\authorfooter{
%% insert punctuation at end of each item
\item
 Zoe Dormuth is a graduate student at the University of Arizona. E-mail:dormuth@email.arizona.edu.
}

%other entries to be set up for journal
%\shortauthortitle{Firstauthor \MakeLowercase{\textit{et al.}}: Paper Title}

%% Abstract section - 5 pts
\abstract{
This project will visualize data about the National Parks. The goal is to create a visualization that will allow the user to see the best time of year to visit each National Park. National Parks are a popular destination for many travelers. Over-crowding has become an issue at many National Parks where permits have been incorporated via a lottery system to mitigate dangerous situations from over-crowding. The visualization will include data on the number of visitors at a National Park, the weather, the cost for transportation and lodging and the activities available at each park at any given time of year. The community will be able to learn the best time of year to visit each National Park for the best experience possible.

The contributions for milestone five included completing the visualization, designing and performing evaluations, summarizing and discussing the results from the evaluations, discussing future work and providing a summary of the visualization. An interview was designed and conducted with two participants. The visualization was completed before performing the interviews. An in-class presentation was given to demonstrate and explain the project. The audience was required to fill out a questionnaire about the visualization discussing the goal, problem the visualization solves, the visualization strengths and where it could be improved. These results were compiled and summarized. 
} % end of abstract

%% Keywords that describe your work. Will show as 'Index Terms' in journal
%% please capitalize first letter and insert punctuation after last keyword
%\keywords{Radiosity, global illumination, constant time}

%% ACM Computing Classification System (CCS). 
%% See <http://www.acm.org/class/1998/> for details.
%% The ``\CCScat'' command takes four arguments.

%\CCScatlist{ % not used in journal version
% \CCScat{K.6.1}{Management of Computing and Information Systems}%
%{Project and People Management}{Life Cycle};
% \CCScat{K.7.m}{The Computing Profession}{Miscellaneous}{Ethics}
%}

%% Uncomment below to include a teaser figure.
%   \teaser{
%   \centering
%   \includegraphics[width=16cm]{CypressView}
%   \caption{In the Clouds: Vancouver from Cypress Mountain.}
%  }

%% Uncomment below to disable the manuscript note
%\renewcommand{\manuscriptnotetxt}{}

%% Copyright space is enabled by default as required by guidelines.
%% It is disabled by the 'review' option or via the following command:
% \nocopyrightspace

\vgtcinsertpkg

%%%%%%%%%%%%%%%%%%%%%%%%%%%%%%%%%%%%%%%%%%%%%%%%%%%%%%%%%%%%%%%%
%%%%%%%%%%%%%%%%%%%%%% START OF THE PAPER %%%%%%%%%%%%%%%%%%%%%%
%%%%%%%%%%%%%%%%%%%%%%%%%%%%%%%%%%%%%%%%%%%%%%%%%%%%%%%%%%%%%%%%%

\begin{document}

%% The ``\maketitle'' command must be the first command after the
%% ``\begin{document}'' command. It prepares and prints the title block.

%% the only exception to this rule is the \firstsection command
\firstsection{Overview} % or "Motivation"

\maketitle

%% Questions for office hours:
%% is the related work similar to PM1 related work with some things added? --> do we write the change in Refinement from Previous Milestone?

Many people have limited time to travel and take a vacation. Popular travel destinations include the National Parks in the United States. A common problem is over-crowding during certain times of the year for many National Parks. It takes away from the natural surroundings of the parks, make activities dangerous and take a toll on the environment.

The goal of this project is to create a visualization that will find the optimal time of year to visit each National Park based on the average number of visitors, weather, cost and type of activity the user would like to do. The user should be able to visit any National Park at the optimal time and avoid crowds while still being able to have a good experience. This visualization problem will combine multiple data sets into one visualization that will provide an overall description of the experience one would expect at a National Park at a specific time of year.

This problem is important because it will allow the user to have better experiences at National Parks while possibly mitigating part of the over-crowding problem the parks are facing. This will entail help preserve the public lands. An decline of accidents may also be seen if this visualization helped mitigate the over-crowding issue. The National Park Service could potentially use this visualization to implement procedures to mitigate the over-crowding issue for certain times of the year. 

% You may want to end this section by summarizing it again as a list:
The work completed in this milestone included:
\begin{itemize}
  \item Added missing parts of visualization that were not completed in Milestone 4
  \item Considered suggestions from presentation feedback and incorporated ones that were simple to implement
  \item Designed evaluation
  \item Conducted evaluation on two participants
  \item Summarized results from evaluation and presentation feedback
  \item Concluded project and stated limitations and future work and evaluations
\end{itemize}

\section{Technical Progress}

The technical pieces of this project were continued in Project Milestone 5. The activities selected in the activities view of the visualization turn the National Park's that have the activity yellow in the visualization. A black background was added to the yellow circles because feedback from the in-class presentation said the yellow circles were difficult to see especially when overlapped. When the activity is unselected, the National Park's circles that had the activity return to the green shade they were originally. The removal of circles was unable to be implemented in a more direct manner. Documentation about implementing the circles on Leaflet with a model-view-controller scheme were found but it would require a refactoring of the code which was unable to be completed in the remaining time for the project. The removal of circles will be communicated to the participant when performing evaluations. The display view does not display Acadia National Park's information when the month selected changes nor when the clearing of the circles on the map occur. The radius of the National Park was scaled to prevent overlapping when popular months are selected in the month selection. The coloring of the circles was modified to accomodate the feedback from the in-class presentation. The scale was changed such that the most popular National Park's coloring is present and not white. 



\section{Evaluation}

\subsection{Interviews}

\subsubsection{Design}
An interview was designed and conducted with two participants. The interview was chosen as an evaluation type because this is an exploratory visualization so the data collected should reflect the users initial thoughts on the visualization. This visualization could lead to creating a recommendation system to solve the original problem proposed at the beginning of the project. This problem was to find the optimal time of year to visit each National Park based on a variety of data. The responses from the interview will be used to understand what the visualization communicates and what information or interactions are missing or misrepresented. This evaluation type allows the interview to be flexible based on the participants response to the visualization and allows for more data to be collected with probing questions. 

The interview is designed as a semi-structured interview where a basic script of pre-determined questions is used and additional questions are added based on the responses of the pre-determined questions. This allows the questions to be more open ended and customized based on the users responses and background. The users backgrounds of the National Park's include more knowledge than an average U.S. resident, however, there exist other users that utilize National Park's more and less than the particpants. The pre-determined questions include background questions about the participants experience with National Park's and why they are interested in National Parks. The visualization is then shown to the user and they can interact with it. Another set of pre-determined questions specifically about the visualization are asked. Next, based on the responses the follow up questions that are not pre-determined are asked to probe any information that is missing or interesting points that came up during the first two parts of the interview. The participants are aware of the structure of the interview and are communicated which section the interview is being performed. 

Although interviews are time consuming, the interviews were only conducted on two participants so the analysis will be shorter than performing the interview on multiple participants. The evaluation of the interview will be aware of subjective results due to the fewer participants and interview type evaluation.

\subsubsection{Results}
The first participant used the pan and zoom feature of the map right away. The map is the main view the participant looks at. They were able to locate their location on the map without any indication. They knew where some National Park's were located but not all. They also were able to know some where in a certain state but not which part of the state. The removal of circles that was not implemented in the visualization was not an issue at all so it did not affect the results. The participant selected the activities they were interested in. They were disappointed it did not list the National Park's that had those activites. A suggestion of a list of the Park's and relevant data would be a good addition to the highlighting. The user had trouble clicking on the box that corresponded to the activity they wanted to select due to awkward spacing. The map view should be centered on the screen. A title of which National Park was selected above the map view would also be useful. Directions on how to use the visualization were also suggested. The price was a highlight of the visualization for the user. 

The second participant was immediately able to find their location on the map. The pan and zoom feature was also used a lot. They didn't know where National Park's are located on a map and suggested a hover list that would show when selected on a map from a list. The user knew the circles encoded population immediately, however, they hestitated about it meaning the size of the park as well. The participant really liked the activity selection and found more information about Gateway NP being a National Park that they have visited before. The map view inspired them to suggest that this should be scaled out to include all the National Parks in the world. 

\subsection{In-class Presentation Feedback}

\subsubsection{Design}
The feedback from the in-class presentation was collected and will be summarized. This was not an evaluation that was designed by the visualization designer, however, this is a type of evaluation that is useful for the visualization. An in-class presentation was given to demonstrate and explain the project. The audience was required to fill out a questionnaire about the visualization discussing the goal of the visualization, problem the visualization solves, the visualization strengths and where it could be improved. These results were compiled and summarized. 

The questionnaire was a quick way to reach out to a large audience of people who knew about visualization to better understand the users initial thoughts of the visualization. The questionnaire collected data about how the visualization could be improved and these results were incorporated into the last technical modifications of the visualization.

\subsubsection{Results}

The main results of the questionnaire resulted in better ways to encode the number of visitors and improve the activity selection. Adding a color scale would help better understand the population encoding. Adding commas to the number of visitors would be easier to read. Adding a juxtaposed view to compare different months and see transitions of number of visitors. Encoding the population of park with area would be good because it is less important how many people are in a very large park compared to a smaller park.

Activity selection could be improved by adding more views or adding glyphs. Aggregating the activity into drop downs might also make this part of the visualization more intuitive. More data could also be encoded into the visualization like temperature and cost. 


\subsection{Discussion}
 A concern for users of the visualization are having them not know where the National Park's are located on a map. This will cause them to have to click on a lot of circles and memorize which National Park's were which circles. A suggestion of having a list and hovering over a National Park and having it displayed on the map was a good idea and would be easily implemented. The map view allowed both users to be able to explore and see more data about parks which they aren't able to do on previous work. 

 The activity selection and information was a plus for the interview participants. These participants are very interested in seeing which activities are available at each National Park. A history of the National Park's selected and their data when choosing activities would be helpful. A more aesthetically pleasing design would please the users more as well which includes centering the map view and adding more views of descriptions of what the interactions entail. 

 The largest take away for encoding the number of visitors at a National Park was correlating it to the size of the park. Normalizing the National Park's area in regards to the number of visitors would be a very important improvement to this visualization. The visualization's motivation is based on popularity of a park so the size and number of visitors should be encoded very prominently. The color scale addition to better understand what the circles were showing was important to the visualization background evaluation but not for the interview participants who do not have a visualization background. Other design techniques from the visualization background evaluations were included like adding glyphs to show weather and activity data. A histogram showing the cost was also a good suggestion from both evaluations. 

 The weather attributes were not used as much. In the interviews the results were that the user already has an idea of when they would like to visit a National Park or is open to anytime. An expert evaluation would provide more insight into this attribute.

 The cost was an important attribute and an explanation of how the cost was calculated would be helpful.

 The most important part of the evaluation was finding how important and useful having all the National Park data in one visualization. This allowed the users to explore the data in an interactive manner and within seconds they discovered information about National Park's that they previously did not know.

\subsection{Limitations}
The limiations included with this evaluation included only interviewing participants who are familiar with National Parks. These users participate in a lot of outdoor activities which are often available at the National Parks. Time was also a restriction and more participants from a range of demographics would benefit the evaluation. The visualization did not include directions on how to use the visualization. Both in the questionnaire and the interviews the visualization was explained in great detail with questions able to be answered. A real-world evaluation of the visualization should include a set of directions such that it is intuitive to the user without having the designer answer questions in real time. Logging would be a good evaluation session to really understand what the user intuitively understands from the visualiztaion without the designer present. This would show how the user interacts naturally with the visualization and which designs are good and where improvement could be included. Including different demographics by setting up an expert evaluation would benefit in the next step of creating a recommendation system. The expert would be able to provide valuable feedback to help understand correlation between the data encoded and might have recommendations as to how to correlate the viewable data. 

\subsection{Future Evaluation}
I would evaluate this project further by interviewing more participants from a wider demographic. This includes an expert, participants with families, different aged participants and different monetary valued participants. This would provide further insight as to what a user finds most important when choosing to visit a National Park. The expert would have a longer evaluation time to gather more data. The interviews could be conducted in 20 minutes with participants from different demographics. A group of 10 participants would be sufficient for the exploratory visualization.

\section{Lessons Learned}

The lessons learned in this visualization included understanding the importance of task taxonomies and being able to abstract a specific problem and design a visualization that will complete those tasks. I also learned that without a lot of knowledge of the technology you are using, implementing some tasks might be harder than originally conceived. This is partly why I had to switch my project from a recommendation system to a more exploratory analysis. Evaluation provided a lot of feedback about how the design techniques worked really well for completing a certain tasks and how to improve the visualization design to better achieve other tasks. The effectiveness and expressiveness principles were used and reflected well in the visualization. The project milestones also allowed a better understanding of how design iterations are important and help the visualization answer the tasks initially proposed. The entire process from finding a problem to seeing if visualization is good for the problem, identifying tasks, using design techniques, writing the code and evaluation can definitely be applied to other visualizations. The collection and analysis of data including different kinds of data can also be applied to other problems and visualizations. This helped identify tasks and understand which data to collect. The aggregated data for this design would be useful in other problems with National Parks because this collection of data in one place did not exist before this project.

\section{Project Summary}

This project is an exploratory analysis of the National Park's in the United States. The National Park's data about number of visitors per month to each National Park, the average high and low temperatures per month, the cost of visiting the National Park, the activities available, the description and location of the National Park's were collected and encoded in a visualization. The project was selected and evaluated to be a visualization problem. Background research, previous visualizations and existing data on the problem were researched and analyzed. A list of tasks using Schneiderman's task taxonomy was used to provide the task and data abstraction. This allowed this specific problem to be generalized to fit existing visualization design techniques. The data was collected from multiple sources and put into one file. This file is the only file to have the specific collection of data from multiple sources. The visualization design was coded using D3 js and Leaflet, a library to encode the geospatial data. The project switched from a recommendation system to an exploratory visualization after the visual design encodings were mapped out by hand. The recommendation system would have required data analysis out of the scope of this project. The visualization was coded for three milestones. An evaluation was conducted using interviews as well as the questionnaire collected after the in-class presentation. These allows the design to be improved with feedback and understand which design techniques solved which tasks the best. Every milestone included a small design iteration in the way the visualization was encoded or a large one for the switch from a recommendation system to an exploratory visualization. 

%\bibliographystyle{abbrv}
\bibliographystyle{abbrv-doi-hyperref}
%%use following if all content of bibtex file should be shown
%\nocite{*}
\bibliography{pm2}
\end{document}

