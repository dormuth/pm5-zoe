\section{Lessons Learned}

The lessons learned in this visualization included understanding the importance of task taxonomies and being able to abstract a specific problem and design a visualization that will complete those tasks. I also learned that without a lot of knowledge of the technology you are using, implementing some tasks might be harder than originally conceived. This is partly why I had to switch my project from a recommendation system to a more exploratory analysis. Evaluation provided a lot of feedback about how the design techniques worked really well for completing a certain tasks and how to improve the visualization design to better achieve other tasks. The effectiveness and expressiveness principles were used and reflected well in the visualization. The project milestones also allowed a better understanding of how design iterations are important and help the visualization answer the tasks initially proposed. The entire process from finding a problem to seeing if visualization is good for the problem, identifying tasks, using design techniques, writing the code and evaluation can definitely be applied to other visualizations. The collection and analysis of data including different kinds of data can also be applied to other problems and visualizations. This helped identify tasks and understand which data to collect. The aggregated data for this design would be useful in other problems with National Parks because this collection of data in one place did not exist before this project.