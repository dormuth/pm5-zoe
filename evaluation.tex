\section{Evaluation}

\subsection{Interviews}

\subsubsection{Design}
An interview was designed and conducted with two participants. The interview was chosen as an evaluation type because this is an exploratory visualization so the data collected should reflect the users initial thoughts on the visualization. This visualization could lead to creating a recommendation system to solve the original problem proposed at the beginning of the project. This problem was to find the optimal time of year to visit each National Park based on a variety of data. The responses from the interview will be used to understand what the visualization communicates and what information or interactions are missing or misrepresented. This evaluation type allows the interview to be flexible based on the participants response to the visualization and allows for more data to be collected with probing questions. 

The interview is designed as a semi-structured interview where a basic script of pre-determined questions is used and additional questions are added based on the responses of the pre-determined questions. This allows the questions to be more open ended and customized based on the users responses and background. The users backgrounds of the National Park's include more knowledge than an average U.S. resident, however, there exist other users that utilize National Park's more and less than the particpants. The pre-determined questions include background questions about the participants experience with National Park's and why they are interested in National Parks. The visualization is then shown to the user and they can interact with it. Another set of pre-determined questions specifically about the visualization are asked. Next, based on the responses the follow up questions that are not pre-determined are asked to probe any information that is missing or interesting points that came up during the first two parts of the interview. The participants are aware of the structure of the interview and are communicated which section the interview is being performed. 

Although interviews are time consuming, the interviews were only conducted on two participants so the analysis will be shorter than performing the interview on multiple participants. The evaluation of the interview will be aware of subjective results due to the fewer participants and interview type evaluation.

\subsubsection{Results}
The first participant used the pan and zoom feature of the map right away. The map is the main view the participant looks at. They were able to locate their location on the map without any indication. They knew where some National Park's were located but not all. They also were able to know some where in a certain state but not which part of the state. The removal of circles that was not implemented in the visualization was not an issue at all so it did not affect the results. The participant selected the activities they were interested in. They were disappointed it did not list the National Park's that had those activites. A suggestion of a list of the Park's and relevant data would be a good addition to the highlighting. The user had trouble clicking on the box that corresponded to the activity they wanted to select due to awkward spacing. The map view should be centered on the screen. A title of which National Park was selected above the map view would also be useful. Directions on how to use the visualization were also suggested. The price was a highlight of the visualization for the user. 

The second participant was immediately able to find their location on the map. The pan and zoom feature was also used a lot. They didn't know where National Park's are located on a map and suggested a hover list that would show when selected on a map from a list. The user knew the circles encoded population immediately, however, they hestitated about it meaning the size of the park as well. The participant really liked the activity selection and found more information about Gateway NP being a National Park that they have visited before. The map view inspired them to suggest that this should be scaled out to include all the National Parks in the world. 

\subsection{In-class Presentation Feedback}

\subsubsection{Design}
The feedback from the in-class presentation was collected and will be summarized. This was not an evaluation that was designed by the visualization designer, however, this is a type of evaluation that is useful for the visualization. An in-class presentation was given to demonstrate and explain the project. The audience was required to fill out a questionnaire about the visualization discussing the goal of the visualization, problem the visualization solves, the visualization strengths and where it could be improved. These results were compiled and summarized. 

The questionnaire was a quick way to reach out to a large audience of people who knew about visualization to better understand the users initial thoughts of the visualization. The questionnaire collected data about how the visualization could be improved and these results were incorporated into the last technical modifications of the visualization.

\subsubsection{Results}

The main results of the questionnaire resulted in better ways to encode the number of visitors and improve the activity selection. Adding a color scale would help better understand the population encoding. Adding commas to the number of visitors would be easier to read. Adding a juxtaposed view to compare different months and see transitions of number of visitors. Encoding the population of park with area would be good because it is less important how many people are in a very large park compared to a smaller park.

Activity selection could be improved by adding more views or adding glyphs. Aggregating the activity into drop downs might also make this part of the visualization more intuitive. More data could also be encoded into the visualization like temperature and cost. 


\subsection{Discussion}
 A concern for users of the visualization are having them not know where the National Park's are located on a map. This will cause them to have to click on a lot of circles and memorize which National Park's were which circles. A suggestion of having a list and hovering over a National Park and having it displayed on the map was a good idea and would be easily implemented. The map view allowed both users to be able to explore and see more data about parks which they aren't able to do on previous work. 

 The activity selection and information was a plus for the interview participants. These participants are very interested in seeing which activities are available at each National Park. A history of the National Park's selected and their data when choosing activities would be helpful. A more aesthetically pleasing design would please the users more as well which includes centering the map view and adding more views of descriptions of what the interactions entail. 

 The largest take away for encoding the number of visitors at a National Park was correlating it to the size of the park. Normalizing the National Park's area in regards to the number of visitors would be a very important improvement to this visualization. The visualization's motivation is based on popularity of a park so the size and number of visitors should be encoded very prominently. The color scale addition to better understand what the circles were showing was important to the visualization background evaluation but not for the interview participants who do not have a visualization background. Other design techniques from the visualization background evaluations were included like adding glyphs to show weather and activity data. A histogram showing the cost was also a good suggestion from both evaluations. 

 The weather attributes were not used as much. In the interviews the results were that the user already has an idea of when they would like to visit a National Park or is open to anytime. An expert evaluation would provide more insight into this attribute.

 The cost was an important attribute and an explanation of how the cost was calculated would be helpful.

 The most important part of the evaluation was finding how important and useful having all the National Park data in one visualization. This allowed the users to explore the data in an interactive manner and within seconds they discovered information about National Park's that they previously did not know.

\subsection{Limitations}
The limiations included with this evaluation included only interviewing participants who are familiar with National Parks. These users participate in a lot of outdoor activities which are often available at the National Parks. Time was also a restriction and more participants from a range of demographics would benefit the evaluation. The visualization did not include directions on how to use the visualization. Both in the questionnaire and the interviews the visualization was explained in great detail with questions able to be answered. A real-world evaluation of the visualization should include a set of directions such that it is intuitive to the user without having the designer answer questions in real time. Logging would be a good evaluation session to really understand what the user intuitively understands from the visualiztaion without the designer present. This would show how the user interacts naturally with the visualization and which designs are good and where improvement could be included. Including different demographics by setting up an expert evaluation would benefit in the next step of creating a recommendation system. The expert would be able to provide valuable feedback to help understand correlation between the data encoded and might have recommendations as to how to correlate the viewable data. 

\subsection{Future Evaluation}
I would evaluate this project further by interviewing more participants from a wider demographic. This includes an expert, participants with families, different aged participants and different monetary valued participants. This would provide further insight as to what a user finds most important when choosing to visit a National Park. The expert would have a longer evaluation time to gather more data. The interviews could be conducted in 20 minutes with participants from different demographics. A group of 10 participants would be sufficient for the exploratory visualization.